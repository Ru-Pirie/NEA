\begin{FlushLeft}
% this is all about reflection on how successful the project has been
% no real excuse to not get full marks
% Section on overall effectiveness of the system, what went well what went wrong, be detialed
% Section on the objcetives, judge how effectivly it has been met also comment on how the solition might be inproved. Can refer back to testing
% End user feedback, the end user needs to feedback to how effective the solution is. Needs to be a critical evaluation. Has to be critasizms, what needs to be improved, but also what went well. D E T A I L
% My insight into their feedback is what gets the marks not their feedback on its own.
% System improvements, explain the outline how you might go about making improvments based on the evaluation and objective completeion and end user feedback. Be honest.
    \section{Evaluation}

    \subsection{Objective Completion}
    \begin{longtable}{| C{2.4cm} | L{12.6cm} |}
        \hline
        \textbf{Objective Number} & \textbf{Completion Details}
        \hline
        1.1 & In its current form my program is able to take an image of a map and convert it successfully into routable data. It accepts most common image formats. \\
        \hline
        1.2 & Should the user select to do so the program will prompt them to enter all of the information about the image that was supplied. Once it collects this information it stores the data in a temporary class in memory and then writes it to a binary save after it has finished processing. \\ 
        \hline 
        1.3 & Once Canny edge detection has been performed upon the image and the set roads have been picked out it, it converts it to a graph inside a windows form to allow the user to pathfind through it. \\
        \hline
        1.4 & When an error occurs the program logs it to a log file allowing the user to go back and see what exactly caused the error. It also shows a large box in the middle of the screen making it clear what just happened. It also contains a small message about how to remedy the error. \\
        \hline
        \hline
        2.1 & Should the user choose to do so the program can save a image of the current edge detection stage. As well as this if the user selects to run a specific way the program will show a preview of the image while it is running.\\
        \hline
        2.2 & Similar to above should the user chose to do so they can have the program pause at every stage and allow them to input different variables and see the effect that this has. \\
        \hline
        2.3 & If the user choses to do so there are 3 presets they can pick from which all they need to do is select then the program will run through all stages autonomously, alternatively they can choose to input values at the very beginning and run from there. \\
        \hline
        2.4 & This is completed should the user select the single threaded option at the Canny edge detection stage. \\
        \hline
        \hline
        3.1 & After the Canny edge detection has finished, a windows form is shown to the user, this contains the output of the edge detection after it has gone through the embossing kernel and pixel filling. \\
        \hline
        3.2 & With the result of the canny edge detection the paths can be detected using a combination of filling and coordinate maths. \\ 
        \hline
        \hline
        4.1 & The user can change the map traversal algorithm in the program settings in order to change the behaviour of the algorithms used, two algorithms are . \\
        \hline
        4.2 & 

        \hline
        \end{longtable}
    \BK

    \subsection

    \subsection{End User Feedback}
    \BK

    \subsection{Reflection on Feedback}

    \subsection {System Improvements}
    I believe that there are several areas to improve upon including some which where not outlined in the end user feedback. \\ \bk

    \paragraph{Graph Traversal} \mbox{}
    When it came down to the ways in which the user could use the map and then traverse it the program did work and achieved the objectives however it is not always what is wanted. I believe that if I could improve the system more I would implement a better system to select the traversal algorithm. As well as this I believe that I would
    
    \BK


\end{flushleft}